\documentclass[12pt]{article}
\usepackage[utf8]{inputenc}
\usepackage[brazil]{babel}
\usepackage[T1]{fontenc}
\usepackage{amsmath, amssymb, array, bm, geometry, booktabs, siunitx, graphicx, colortbl, parskip, xcolor}
\usepackage{listings}
\usepackage{color}
\usepackage{float}
\usepackage{fancyhdr}
\usepackage{titlesec}
\usepackage{hyperref}
\usepackage{listings}

\setlength{\headheight}{14.5pt}
\addtolength{\topmargin}{-2.5pt}
\geometry{a4paper, margin=2.5cm}

\definecolor{codegray}{gray}{0.9}
\lstset{
    backgroundcolor=\color{codegray},
    basicstyle=\ttfamily\footnotesize,
    frame=single,
    breaklines=true,
    captionpos=b,
    numbers=left,
    numberstyle=\tiny,
    language=Python
}

% Cabeçalho
\pagestyle{fancy}
\fancyhf{}
\rhead{UnB}
\lhead{Departamento de Ci\^encias Mec\^anicas}
\cfoot{\thepage}

\titleformat{\section}{\large\bfseries}{\thesection}{1em}{}

\begin{document}

% Capa
\begin{titlepage}
    \centering
    \includegraphics[width=12cm]{img/unb_bandeira.png} \\
    \vspace{1cm}
    \textsc{\Large Universidade de Bras\'ilia} \\
    \textsc{Departamento de Ciências Mec\^anicas} \\
    \textsc{Programa de P\'os-Gradua\c{c}\~ao} \\
    \vfill
    {\Large\bfseries Atividade 5} \\
    \vspace{0.5cm}
    {\Large\bfseries Métodos Numéricos de Otimização} \\
    {\Large\bfseries SIMPLEX} \\
    \vspace{0.5cm}
    \textbf{Disciplina: M\'etodos Num\'ericos} \\
    Professor: Dr. Rafael Gabler Gontijo \\
    \vfill
    \textbf{Aluno: Eng. Lucas Wanick — Mestrando em Engenharia Mec\^anica} \\
    \vspace{0.5cm}
        \today \\
\end{titlepage}

\section*{Enunciado 1 - Interpretação da Tabela Simplex}

\section*{Resolução do Enunciado 1 - Método Simplex}

Nesta primeira parte da atividade, foi solicitado resolver um problema de maximização via Método Simplex. A seguir, é apresentada as três tabelas principais que compõem a sequência de iterações do método.

\subsection*{Iteração 0 - Tabela Inicial}
\begin{center}
\renewcommand{\arraystretch}{1.3}
\begin{tabular}{c|ccccccc|c|c}
\textbf{Básicas} & Z & $x_1$ & $x_2$ & $s_1$ & $s_2$ & $s_3$ & $s_4$ & \textbf{Solução} & \textbf{Interseção} \\
\hline
Z & 1   & 0 & -55 & 0 & 15 & 0 & 0 & 1200 & -- \\
$s_1$ & 0 & 0 & 5{,}4 & 1 & -0{,}7 & 0 & 0 & 21 & 3{,}889 \\
$x_1$ & 0 & 1 & 0{,}8 & 0 & 0{,}1 & 0 & 0 & 8 & 10 \\
$s_3$ & 0 & 0 & -0{,}8 & 0 & -0{,}1 & 1 & 0 & 1 & -1{,}25 \\
$s_4$ & 0 & 0 & 1 & 0 & 0 & 0 & 1 & 6 & 6 \\
\end{tabular}
\end{center}

\noindent
Nesta tabela, a variável $x_2$ possui o menor coeficiente negativo na função objetivo ($-55$), logo é selecionada para entrar na base. Calculamos as razões (Solução / Coluna de $x_2$) para as linhas com coeficientes positivos em $x_2$, obtendo:

\[
\text{Razões:} \quad \frac{21}{5{,}4} \approx 3{,}889, \quad \frac{8}{0{,}8} = 10
\]

A menor razão ocorre na linha de $s_1$, que é então a variável que sai da base.

\subsection*{Iteração 1 - Após entrada de $x_2$ e saída de $s_1$}
\begin{center}
\renewcommand{\arraystretch}{1.3}
\begin{tabular}{c|ccccccc|c|c}
\textbf{Básicas} & Z & $x_1$ & $x_2$ & $s_1$ & $s_2$ & $s_3$ & $s_4$ & \textbf{Solução} & \textbf{Interseção} \\
\hline
Z  & 1 & 0 & -55 & 0 & 15 & 0 & 0 & 1200 & ? \\
$x_2$ & 0 & 0 & 1 & 0{,}1852 & -0{,}1296 & 0 & 0 & 3{,}889 & ? \\
$x_1$ & 0 & 1 & 0{,}8 & 0 & 0{,}1 & 0 & 0 & 8 & ? \\
$s_3$ & 0 & 0 & -0{,}8 & 0 & -0{,}1 & 1 & 0 & 1 & ? \\
$s_4$ & 0 & 0 & 1 & 0 & 0 & 0 & 1 & 6 & ? \\
\end{tabular}
\end{center}

\noindent
Nesta iteração, todas as variáveis da função objetivo (linha Z) têm coeficientes não negativos. Portanto, atingimos a solução ótima.

\subsection*{Iteração Final - Solução Ótima}
\begin{center}
\renewcommand{\arraystretch}{1.3}
\begin{tabular}{c|ccccccc|c}
\textbf{Básicas} & Z & $x_1$ & $x_2$ & $s_1$ & $s_2$ & $s_3$ & $s_4$ & \textbf{Solução} \\
\hline
Z & 1  & 0 & 0 & 10{,}186 & 7{,}872 & 0 & 0 & 1413{,}895 \\
$x_2$ & 0 & 0 & 1 & 0{,}1852 & -0{,}1296 & 0 & 0 & 3{,}889 \\
$x_1$ & 0 & 1 & 0 & -0{,}1482 & 0{,}2037 & 0 & 0 & 4{,}8888 \\
$s_3$ & 0 & 0 & 0 & 0 & -0{,}2037 & 1 & 0 & 4{,}111 \\
$s_4$ & 0 & 0 & 0 & -0{,}1852 & 0{,}1296 & 0 & 1 & 2{,}111 \\
\end{tabular}
\end{center}

\subsection*{Critério de Parada}

O método Simplex atinge o ponto ótimo quando todos os coeficientes da linha da função objetivo (Z) se tornam nulos ou positivos. Isso significa que não há mais direção viável que permita o aumento da função objetivo, e, portanto, a solução encontrada é a melhor possível dentro da região factível.

\noindent
Solução ótima final: $\boxed{Z = 1413{,}895}$ com $x_1 = 4{,}8888$, $x_2 = 3{,}889$.


\section*{Enunciado 2}
\section*{Introdução}

Faremos a modelagem e solução de um problema de programação linear voltado à distribuição ótima da geração de energia elétrica a partir de quatro fontes renováveis no contexto brasileiro: solar fotovoltaica, eólica onshore, biomassa e pequenas centrais hidrelétricas (PCH). O objetivo é minimizar o custo total de geração (LCOE) respeitando restrições operacionais e ambientais realistas, com base em dados extraídos de relatórios técnicos da EPE (Empresa de Pesquisa Energética), IEA (International Energy Agency), IPCC (Intergovernmental Panel on Climate Change), e documentos oficiais brasileiros como o Plano Decenal de Expansão de Energia (PDE).

A modelagem considera os custos nivelados de geração elétrica (em R\$/MWh), as emissões de CO$_2$ equivalente por MWh com base em Análise do Ciclo de Vida (ACV), os fatores de capacidade médios e a produção semanal típica por MW instalado. Tais parâmetros foram utilizados para construir uma função objetivo e um conjunto de restrições coerentes com a realidade brasileira.

A ferramenta computacional utilizada para a resolução do problema foi a biblioteca \texttt{scipy.optimize.linprog}, com o solver HiGHS, que permite a aplicação eficiente de algoritmos simplex e interior point em problemas lineares de média e grande escala. Os resultados obtidos fornecem uma estratégia de despacho energético ótima que atende à demanda semanal de 1000 MWh, respeitando o teto de emissão de 30.000 kgCO$_2$-eq e garantindo participação expressiva de fontes renováveis de baixa emissão.

As fontes e dados utilizados para embasamento incluem:

\begin{itemize}
    \item EPE - Caderno de Preços de Referência para Geração (2023);
    \item IPCC AR5 e relatórios de ciclo de vida energético (2014 e 2021);
    \item IEA World Energy Outlook (edições 2022 e 2023);
    \item Plano Decenal de Expansão Energética 2032 (PDE 2032);
    \item NDC Brasil (Contribuição Nacionalmente Determinada - compromisso com a neutralidade de carbono até 2050).
\end{itemize}

\section*{Formulação Matemática do Problema}

Sejam as variáveis de decisão:

\begin{itemize}
  \item $x_1$: quantidade de energia solar gerada (em MWh/semana)
  \item $x_2$: quantidade de energia eólica gerada (em MWh/semana)
  \item $x_3$: quantidade de energia de biomassa gerada (em MWh/semana)
  \item $x_4$: quantidade de energia gerada por PCH (em MWh/semana)
\end{itemize}

A função objetivo representa o custo total de geração semanal:

\[
\text{Min } Z = 149x_1 + 129x_2 + 247x_3 + 193x_4
\]

Sujeita às seguintes restrições:

\begin{align*}
\text{(R1) Demanda mínima de energia:} &\quad x_1 + x_2 + x_3 + x_4 \geq 1000 \\
\text{(R2) Limite solar:} &\quad x_1 \leq 300 \\
\text{(R3) Limite eólica:} &\quad x_2 \leq 400 \\
\text{(R4) Limite biomassa:} &\quad x_3 \leq 500 \\
\text{(R5) Limite PCH:} &\quad x_4 \leq 450 \\
\text{(R6) Emissão de CO$_2$-eq:} &\quad 48x_1 + 11x_2 + 230x_3 + 24x_4 \leq 30000 \\
\text{(R7) Meta de renováveis limpas:} &\quad x_1 + x_2 \geq 350 \\
\text{(R8) Não negatividade:} &\quad x_i \geq 0, \text{ para } i = 1,2,3,4
\end{align*}

\subsection*{Forma padrão para o solver (linprog)}

A função \texttt{linprog} exige que todas as restrições estejam na forma $A_{ub}x \leq b_{ub}$. Assim, as restrições R1 e R7 são multiplicadas por $-1$:

\begin{align*}
\text{(R1*)} &\quad -x_1 - x_2 - x_3 - x_4 \leq -1000 \\
\text{(R7*)} &\quad -x_1 - x_2 \leq -350
\end{align*}

\section*{Sistema Linear}

\textbf{Função objetivo (vetor c):}
\[
c = \begin{bmatrix}
149 & 129 & 247 & 193
\end{bmatrix}
\]

\textbf{Matriz das restrições (A\_ub):}
\[
A_{ub} =
\begin{bmatrix}
 -1 & -1 & -1 & -1 \\
  1 &  0 &  0 &  0 \\
  0 &  1 &  0 &  0 \\
  0 &  0 &  1 &  0 \\
  0 &  0 &  0 &  1 \\
 48 & 11 &230 & 24 \\
 -1 & -1 &  0 &  0
\end{bmatrix}
\quad
b_{ub} =
\begin{bmatrix}
-1000 \\ 300 \\ 400 \\ 500 \\ 450 \\ 30000 \\ -350
\end{bmatrix}
\]


\section*{Implementação Computacional (Python)}

\begin{verbatim}
import numpy as np
from scipy.optimize import linprog

c = np.array([149, 129, 247, 193])
A_ub = np.array([
    [-1, -1, -1, -1],
    [ 1,  0,  0,  0],
    [ 0,  1,  0,  0],
    [ 0,  0,  1,  0],
    [ 0,  0,  0,  1],
    [48, 11, 230, 24],
    [-1, -1, 0, 0]
])
b_ub = np.array([
    -1000,
    300,
    400,
    500,
    450,
    30000,
    -350
])

res = linprog(c, A_ub=A_ub, b_ub=b_ub, method='highs')
print(res)
\end{verbatim}

\section*{Resultados Obtidos}

\begin{itemize}
    \item Status da Otimização: \textbf{Ótima (convergência)}.
    \item Custo Total Mínimo: \textbf{R\$ 154.200,00}
    \item Geração ótima por fonte:
    \begin{itemize}
        \item Solar: $x_1 = 300$ MWh
        \item Eólica: $x_2 = 400$ MWh
        \item Biomassa: $x_3 = 0$ MWh
        \item PCH: $x_4 = 300$ MWh
    \end{itemize}
    \item Emissões totais de CO$_2$-eq:
    \[
    48 \cdot 300 + 11 \cdot 400 + 230 \cdot 0 + 24 \cdot 300 = 26.000 \text{ kg CO$_2$-eq.} \leq 30.000
    \]
    \item Total gerado: $300 + 400 + 0 + 300 = 1000$ MWh (atende à demanda) $\textbf{(Atendido)}$
    \item Meta de renováveis limpas: $x_1 + x_2 = 700$ MWh $\geq$ 350 $\quad$ $\textbf{(Atendido)}$
\end{itemize}

\end{document}